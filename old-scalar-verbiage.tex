\subsection{Scalar Adjustment}
First, a scalar is required because the relative value of status and autonomy is not inherently fixed in a social system. We might imagine situations where one desires more status at the expense of autonomy if the rewards to status are higher, or values autonomy more highly if the rewards to status are low. There cannot be an assumed unadjusted equivalence between the resulting eigenvectors of the two terms for this theoretical level.

Additionally, due to the hierarchical nature of networks, the average constraint in the network is greater than the average status. The high average constraint of their neighbors is often much larger than the impact of status. The result is that, at the level of the entire network, those who are deeply embedded among the most elite, without a basis of supporters, are very indistinct when considering the high constraints placed on them by their high-status alters. Thus, the calculated results without a scalar may fail to capture the appropriate value of status compared to the constraint.

While an arbitrary scalar is possible, it can yield erroneous results when applied to empirical networks. However, a scalar can be derived from the network itself due to the field positional interplay between status and constraint. Imagining two actors of equal status and personal constraint, one in an averagely more constrained system and the other in a less averagely constrained system, with average status otherwise equal. In such a case, the status of an actor in the more constrained system has a greater impact on their social distinction than in the less constrained system. With high constraint, status takes on greater importance in determining social distinction because it compels deference. In other words, it is the difference between a hierarchical system with and without constraints. Status with a weak constraint means a weaker compulsion of deference. The structure of constraint has a reverberative impact on social distinction beyond its contribution.

With this social understanding of how constraint interacts with the transmission of status, an endogenous scalar can be derived from the network. The average constraint of actors within the network divided by the average status of actors in the network creates a scalar that reflects the relationship between status and constraint. 

\[S=\frac{\frac{1}{n_{j=i}}\sum_{{j}\in{A},i\neq{j}}x_{j}}{\overline{x_i}} \]

Under conditions of high average constraint and low average status, the scalar is large, indicating the significant role of status in determining social distinction. Under conditions where the average constraint is low and the average status is high, the scalar is weaker, reflecting a system in which status is less important in determining social distinction. Under very large-scale conditions, the averages of status and constraint should trend towards 1, weakening the necessity of the scalar. However, under constrained conditions, particularly in elite networks with missing data on the social scaffolding that created the network elites, the scalar is more important for explaining action because the range of eigenvector network measures is otherwise artificially expanded due to the missing data.  

Without a scalar, networks with any degree of hierarchy see the average constraint overwhelm the average status because many low-status nodes are tied to high-status nodes. The advantage of this scalar is to ensure that the average scalar status in the network equals the average constraint in the network. Thus, total status is equal to total constraint, and from there it becomes a matter of distribution among members. From there, however, it enables comparability between networks that would otherwise be impossible. With the understanding that we have, through the scalar, adjusted for a network where one's combination of status and constraint is better or worse, networks can be compared so long as the unobserved social context within the network remains stable. 