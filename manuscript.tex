\documentclass[12pt]{article}
\usepackage[left=1.15in, right=1.15in, top=1in, bottom=1.5in]{geometry}
\usepackage{natbib}
\usepackage{verbatim}
\usepackage{setspace}
\usepackage{booktabs}
\usepackage{caption}
\usepackage{graphicx}
\usepackage{subcaption}
\bibpunct[: ]{(}{)}{;}{a}{}{,} 
\usepackage{authblk}
\renewcommand\Authfont{\normalsize}
\renewcommand\Affilfont{\footnotesize}

\begin{document}
\title{The Network Origins of Social Distinction}
\author[1]{Author Isaac Jilbert\thanks{issac.jilbert@gmail.com}}
\author[1]{Omar Lizardo\thanks{olizardo@soc.ucla.edu}}
\affil[1]{Department of Sociology, UCLA}

\renewcommand\Authands{ and }

\date{\normalsize \today}	
\maketitle

\newpage
\begin{abstract}	
\end{abstract}

\newpage
\section{What is Distinction?}

Distinction, as the expression of status and uniqueness, is a social end in itself. While cultural distinction is empirically a matter of the deference given to symbolic forms of differentiation \citep{bourdieu84}, a generalized sense of social distinction---that which emerges from a network of concrete social relations---has been less precisely conceptualized, and measurement strategies are naturally lacking. Instead, social distinction is often measured as the shadow of cultural distinction or collapsed into the concept of status more generally, which ignores questions of how network constraint impacts the capacity of actors to be distinct. 

For instance, Pierre Bourdieu’s idea of distinction focuses on how individuals adopt and inherit particular cultural practices that demarcate class membership and how these markers become constant signifiers of the underlying social hierarchy. From this perspective, cultural practices reflect past and present relationships and facilitate future connections \citep{bourdieu84}. Here, the focus is placed on cultural distinction, which reflects one’s position in an interconnected system, sometimes referred to as a ``field.'' Distinction is a mark of status where those participating in the same cultural forms share similar social standing in a hierarchy. 

In developing a broader understanding and measure of social distinction,  actor-level incentives and structural impediments to action become apparent. When we understand the underlying motivation actors have to proactively reshape their network connections so as to achieve more distinct positions, historical patterns of network change can be better accounted for and sometimes even properly retrodicted. 

In addition to being a mark of differentiation from others outside the relevant milieu (as in the symbolic status model), for an actor, distinction can also be a question of uniqueness within a delimited set of relationships within which the actor is embedded. Even within a group that shares cultural practices, there are rewards to being differentiated within the category. Complete cultural conformity is a trade-off, where one is completely enmeshed within the group's expectations. Here, the failure to be unique means that one is incapable of asserting autonomy when action is required. For an actor, achieving cultural distinction thus becomes an optimization problem, trading off connectedness for uniqueness.  

In this respect, rather than distinction being a phenomenon about culture and network positioning, it is a more general expression of the interaction between status and uniqueness. When it comes to questions of political power rather than cultural cache, network status is another name for \textit{influence} and uniqueness for \textit{autonomy} in a social topology. An actor with high capacity is well connected to others who can be mobilized for action. An actor with high autonomy is not constrained by these connections or forced into action by them \citep{burt76}. We often use context-specific words to denote position in the hierarchy and ability to act within that hierarchy.

\subsection{A Relational Understanding of Distinction}
To simplify a bit for argument's sake, as we have intimated, the first aspect of social distinction generally is \textit{status}. Status is the internal stratification of all members of the network, along with connections of deference and obligation. In the case of politics, status is the ability to access those who themselves have higher levels of access, with those most central to the access network having the ability to make decisions. This can be captured traditionally by measures of status such as eigenvector centrality \citep{bonacich72}. Status is higher for actors when the people they are connected to are themselves important, and accessibility is just a political reframing of the context. The internal stratification of individuals is determined by who has greater connections to those who have greater connections, with those with few connections to other unimportant people being at the bottom of the hierarchy (Bonacich).  

The second aspect of social distinction is \textit{autonomy}, which is the absence of network constraint. Being structurally unique minimizes the social constraints placed on an individual. The tighter the interconnections of the group, the more its pressures constrain the individual. In politics, this aspect of distinction takes the form of constraint. When an actor is one of many densely interconnected individuals, their ability to act contrary to their networked group is greatly weakened. However, when one is only a weakly dependent and connected member of a group, they have a greater ability to operate independently. Their ability to act is a source of distinction as they are able to take independent action. The question of structural uniqueness and autonomy is the same underlying consideration as captured by measures of path centrality. The more critical one is to bridge connections between individuals within a network, the greater one's importance, and the lower their constraint \citep{burt92}. 

\citet[1263-4]{padgett_ansell94} were the first to bring attention to how the cohesion of those around a broker can structure their own decisions. Cosimo de Medici, acting as a broker between two antagonistic groups, was constrained in the actions he could take because strong actions favoring one side would upset the other side. However, such a constraint was not an issue because Medici’s interests were in perpetuating his central role as broker, and, as the only source of interest, he was able to maintain the position. With regard to conceptualizing distinction, the theory of robust action puts forward that those at the center of political networks, as long as their interests derive solely from maintaining that position, engage in moderate strategies that do not overly antagonize either side. The decisions derive from ensuring the continuation of their position at the center of the network. 

\begin{figure}
    \captionsetup[subfigure]{font=footnotesize,labelfont=footnotesize}
    \centering
     \begin{subfigure}[b]{0.45\textwidth}
        \includegraphics[width=1.0\textwidth]{Plots/star.png}
            \caption{Star Graph}
            \label{fig:star}
    \end{subfigure}
     \begin{subfigure}[b]{0.45\textwidth}
        \includegraphics[width=1.0\textwidth]{Plots/sf.png}
            \caption{Structural Fold}
            \label{fig:sf}
    \end{subfigure}
     \begin{subfigure}[b]{0.45\textwidth}
        \includegraphics[width=1.0\textwidth]{Plots/inter.png}
            \caption{Interconnected Clusters}
            \label{fig:inter}
    \end{subfigure}
     \begin{subfigure}[b]{0.45\textwidth}
        \includegraphics[width=1.0\textwidth]{Plots/intra.png}
            \caption{Intraconnected clusters}
            \label{fig:intra}
    \end{subfigure}
    \caption{}
    \label{fig:toys}
\end{figure}

The interaction of influence and autonomy combines to create an actor’s overall distinction in the network. The combination of influence and autonomy results in capacity, and the social reward for this capacity is distinction. An increase in one aspect can often, but not necessarily, lead to a decrease in the other. For example, a connection with someone with high accessibility can lead to a decrease in one’s autonomy, and new access comes with new constraints. However, the trade-off can maximize one’s overall distinction, so there are gains from the connection. Without such gains, the connection would not be formed.

\begin{figure}
    \captionsetup[subfigure]{font=footnotesize,labelfont=footnotesize}
    \centering
     \begin{subfigure}[b]{0.6\textwidth}
        \includegraphics[width=1.0\textwidth]{Plots/medici.png}
            \caption{Medici Network}
            \label{fig:medici}
    \end{subfigure} \\
     \begin{subfigure}[b]{0.6\textwidth}
        \includegraphics[width=1.0\textwidth]{Plots/nomedici.png}
            \caption{Medici Network without the Medici}
            \label{fig:nomedici}
    \end{subfigure}
    \caption{}
    \label{fig:medici}
\end{figure}

\begin{table}
\centering
\caption{\label{tab:star}Distinction centrality scores for the star graph.}
\centering
\begin{tabular}[t]{lrrrr}
\toprule
Nodes & $\beta_i$ & $s_i$ & $\kappa_i$ & $\alpha$\\
\midrule
1 & 1.230 & 0.707 & 0.000 & 1.739\\
2, 3, 4, 5, 6, 7 & -0.205 & 0.289 & 0.707 & 1.739\\
\bottomrule
\end{tabular}
\end{table}

\begin{table}
\centering
\caption{\label{tab:sf}Distinction centrality scores for the structural fold graph.}
\centering
\begin{tabular}[t]{lrrrr}
\toprule
Nodes & $\beta_i$ & $s_i$ & $\kappa_i$ & $\alpha$\\
\midrule
1 & 0.026 & 1.000 & 1.000 & 1.026\\
2, 3, 4, 5, 6, 7 & -0.004 & 0.608 & 0.628 & 1.026\\
\bottomrule
\end{tabular}
\end{table}

\begin{table}
\centering
\caption{\label{tab:tab:intra}}
\centering
\begin{tabular}[t]{rrrrrr}
\toprule
Node & Distinction & SDistinction & Status & Constraint & Scalar\\
\midrule
1 & 0.195 & 0.369 & 1.000 & 0.805 & 1.174\\
2 & -0.257 & -0.170 & 0.500 & 0.757 & 1.174\\
3 & -0.257 & -0.170 & 0.500 & 0.757 & 1.174\\
4 & 0.048 & 0.155 & 0.618 & 0.570 & 1.174\\
5 & 0.048 & 0.155 & 0.618 & 0.570 & 1.174\\
6 & -0.257 & -0.170 & 0.500 & 0.757 & 1.174\\
7 & -0.257 & -0.170 & 0.500 & 0.757 & 1.174\\
\bottomrule
\end{tabular}
\end{table}

\begin{table}
\centering
\caption{\label{tab:tab:inter}}
\centering
\begin{tabular}[t]{rrrrrr}
\toprule
Node & Distinction & SDistinction & Status & Constraint & Scalar\\
\midrule
1 & 0.251 & 0.361 & 1.000 & 0.749 & 1.11\\
2, 7 & -0.047 & 0.025 & 0.654 & 0.701 & 1.11\\
3, 6 & -0.299 & -0.248 & 0.465 & 0.765 & 1.11\\
4, 5 & -0.025 & 0.042 & 0.612 & 0.637 & 1.11\\
\bottomrule
\end{tabular}
\end{table}



\newpage


\newpage
\bibliography{networks}
\bibliographystyle{apalike}

\end{document}