\documentclass[12pt]{article}
\usepackage[left=1.15in, right=1.15in, top=1in, bottom=1.5in]{geometry}
\usepackage{natbib}
\usepackage{verbatim}
\usepackage{setspace}
\usepackage{booktabs}
\usepackage{caption}
\usepackage{graphicx}
\usepackage{subcaption}
\usepackage{listings}
\usepackage{courier}
\lstset{basicstyle=\footnotesize\ttfamily,breaklines=true}
\lstset{framextopmargin=50pt,frame=bottomline}
\usepackage[hidelinks]{hyperref}
\bibpunct[: ]{(}{)}{;}{a}{}{,} 
\usepackage{authblk}
\renewcommand\Authfont{\normalsize}
\renewcommand\Affilfont{\footnotesize}

\begin{document}
\title{Network Origins of Social Distinction}
\author[1]{Author Isaac Jilbert\thanks{isaac.jilbert@gmail.com}}
\author[1]{Omar Lizardo\thanks{olizardo@soc.ucla.edu}}
\affil[1]{Department of Sociology, UCLA}

\renewcommand\Authands{ and }

\date{\normalsize \today}	
\maketitle

\newpage
\begin{abstract}	
This paper introduces the concept of \textit{distinction centrality}, developing a network metric designed to capture it. The metric is theoretically inspired by Bourdieu's theory of agonistic competition for distinction in social fields, where other actors and their actions provide opportunities and constraints to maximize social positioning. The basic idea behind the network metric of distinction is to subtract the \textit{autonomy} of each actor within the network from their \textit{constraint}, such that highly autonomous (and therefore less constrained) actors have more distinction. Autonomy is given by the focal actor's Eigenvector Centrality score (recording connections to well-placed others), and constraint is defined by the average Eigenvector Centrality score of each actor's neighbors in a hypothetical network where the focal actor is removed (indicating how others would benefit from that actor's exiting the network). A scalar is then applied to create mathematical and sociological equivalence between autonomy and constraint. The resulting node-level measures create a measure of \textit{distinction}, where the uniqueness of an actor's relations is more heavily rewarded than under the standard Eigenvector Centrality score. We analyze two classic datasets, Padgett and Ansell's \citeyearpar{padgett1993robust} Medici Network and Zachary's \citeyearpar{zachary1977information} Karate Club, to demonstrate that the observed actions of the agents involved are consistent with the maximization of distinction. We close by outlining how the resulting distinction measure can be used to both forecast network transformations and anticipate competitive changes resulting from external network shocks.    
\end{abstract}

\newpage
\section{Distinction Beyond Taste}
Popular within cultural sociology, the concept of distinction has generated a rich body of literature aimed at explaining cultural differentiation within social fields. From this perspective, the pursuit of distinction, as the expression of an actor's status and uniqueness, is an end in itself, providing the basic motivation for action \citep{martin2003field}. The question of cultural distinction is often treated empirically as a matter of the deference given to symbolic forms of differentiation enacted by designated institutional authorities \citep{bourdieu1984distinction}. 

In this last respect, the more encompassing notion of \textit{social} distinction, implying an actor's embeddedness in a relational field \citep{erikson2013formalist-400, fuhse2021social-9cb}, is often measured as the shadow counterpart of cultural distinction or collapsed into the concept of status more generally \citep{podolny2010status-1a6}. This approach overlooks questions of how network constraints produced by other agents impact the capacity of actors to be distinct. A generalized, measured ordering of social distinction that is derivative from a network of concrete social relations would give Bourdieu's generalized conception of distinction a broader methodological base from which to conduct empirical analysis. 

\subsection{What is Distinction?}
Bourdieu was primarily interested in the competition among actors for positions within the hierarchy implied by the field \citep{anheier1995forms, bourdieu1993field-8ad}. Bourdieu defined the field as a social space where actors operate within socially negotiated boundaries that arbitrate relevant competitive claims over position \citep{bourdieu1996rules-393}. In interplay with the field, individuals develop a habitus of embodied dispositions, expressed as cognitive ease with particular types of settings, objects, and performances. An actor's position in the field influences their habitus. Similarly, their habitus determines their future possibilities within the field, while also shaping the opportunities available to others in the field. Actors compete for positions in the field via various forms of capital or resources that are socially agreed upon as desirable within a field \citep{martin2003field}, thereby facilitating the formation of connections that both order and reflect the hierarchy of the field. The interaction of the desirability and uniqueness of the capitals possessed by an individual determines their place in the field's positional hierarchy.   

Bourdieu \citeyearpar{bourdieu1984distinction} most famously used the conceptualization of distinction to explain the persistence of the relationship between class and cultural hierarchies. Cultural elites in France distinguished themselves from their inferiors by adopting differentiated cultural practices that had undergone a process of consecration, which set them apart from the common. There is a strong correlation between those in the position of cultural and economic elites. Still, such a correlation is not guaranteed at the individual level, even if it is relevant at the field level.   

While largely a concept used by those who research cultural practices, the pursuit of distinction is a general expression of the interaction between status and uniqueness within a field of positions \citep{anheier1995forms}. For example, while cultural practices may be a marker of difference, the absence of association with others outside the milieu is also a matter of agency within the milieu, a question of the constraints imposed by the set of relationships within which the actor is embedded. For Bourdieu, the struggle for position in a field was a product of a field of action and a field of forces \citep{martin2003field}. The field of action refers to the agency of actors actively advancing their interests. In contrast, the field of forces describes the counteracting pressures faced by the other participants whose interests would be upset by positional transformations.   

In relation to theory, network status is another name for \textit{influence} \citep{friedkin1991theoretical}, and uniqueness for \textit{autonomy} in a social environment \citep{burt1980autonomy}. Translated to network terms, Bourdieu's field of action can be understood as the status hierarchy of the network, while the field of struggle is reflective of the constraints neighbors put on an actor. An actor with high influence is well-connected to others who can be mobilized for effective action. An actor with high autonomy is not constrained by their pattern connections or forced into action by them \citep{burt76}. We often use context-specific words to denote position in the hierarchy and ability to act within that hierarchy. Nevertheless, we use ``status" and ``autonomy" below as general terms to describe the ability of actors to take action versus the level of imposition from the forces of constraint.

In developing a broader understanding and measure of social distinction, actor-level incentives and structural impediments to action become apparent. When we understand the underlying motivation actors have to proactively reshape their network connections so as to achieve more distinct positions, historical patterns of network change can be better accounted for and sometimes even properly forecasted. 

\subsection{Position as the Relational Basis for a Distinction Measure}
Bourdieu's separation of distinction into habitus and position is not a sociological problem at all, but rather creates the analytical problem of how an actor's network position generates dispositions and how these dispositions create opportunities for network mobility. Both dispositions and positions are fundamentally relational outcomes in sociological terms; however, when creating a network measure of distinction, only the relational basis of position can be considered. 

However, to the extent that position and habitus reflect each other through their symbiotic fabrication, the ability to empirically capture only one of the two components is sufficiently accurate to measure a social system's distinction hierarchy with confidence over a short window of time. Actors should not hold a network position that does not align with their habitus, and their current habitus and social position determine their ability to radically change positions. If the transformation of habitus is a stepwise interplay with positional movement that unfolds over time, the measure becomes less informative if it captures networks over extended periods. However, even over long periods, this theoretical collapse, which permits a network-measured methodological shortcut, may still be adequate in a moderately stable social environment. 

To establish a firm theoretical foundation for the measure of distinction, it is first essential to make explicit the relationship between empirical data and social theory that forms the basis of social network analysis. Network measures are fundamentally based on the idea that researchers are capturing a relationship between two objects in some way. A tie between two nodes captures, in some way, their connection, however it is defined. Such connections can be broadly classified as mutual, in the case where the relationship quality is equal in both directions, or directional, in the case where the qualities are unequal. In the absence of direct measurement of qualities, however, many networks are measured as consisting of mutual ties. However, even in the case of networks composed of only mutual ties, hierarchy can still be assessed from the pattern of relationships. In such networks, we can still rank the direction of power and deference through the absence of ties between lower and higher nodes. For example, in an organizational pyramid, we can detect the CEO by observing the presence of ties with vice presidents and the absence of ties with middle management and production workers. We can do this through the measurement of a node's network status. 

The first aspect of social distinction is \textit{status}. Status refers to the internal stratification of all network members, along with the connections of deference and obligation. In the case of politics, status refers to the ability to access those who themselves have higher levels of access, with those most central to the access network having the authority to make decisions. This can be captured traditionally by measures of status such as eigenvector centrality \citep{bonacich1972norms}. Status is higher for actors when the people they are connected to are themselves influential, and accessibility is merely a political reframe of the context. The internal stratification of individuals is determined by who has greater connections to those with even greater connections, with those having few connections to other important people being at the bottom of the hierarchy \citep{bonacich1972norms}. 

The second aspect of social distinction is \textit{autonomy}, which is the absence of network constraint \citep{burt1980autonomy}. Being structurally unique minimizes the social constraints placed on an individual. The tighter the interconnections of the group, the more its pressures constrain the individual. In politics, this aspect of distinction takes the form of constraint. When an actor is one of many densely interconnected individuals, their ability to act contrary to their networked group is greatly weakened. However, when one is only a weakly dependent and connected member of a group, they have a greater ability to operate independently. Their ability to act is a source of distinction as they are able to take independent action. The question of structural uniqueness and autonomy is the same underlying consideration as captured by measures of path centrality. The more critical one is to bridge connections between individuals within a network, the greater one's importance, and the lower their constraint \citep{burt1995structural-5e8}. 

\citet[1263-4]{padgett1993robust} were the first to bring attention to how the cohesion of those around a broker can structure their own decisions. Cosimo de Medici, acting as a broker between two antagonistic groups, was constrained in the actions he could take because strong actions favoring one side would upset the other side. However, such a constraint was not an issue because Medici’s interests were in perpetuating his central role as broker, and, as the only source of interest, he was able to maintain the position. With regard to conceptualizing distinction, the theory of robust action puts forward that those at the center of political networks, as long as their interests derive solely from maintaining that position, engage in moderate strategies that do not overly antagonize either side. The decisions derive from ensuring the continuation of their position at the center of the network. 

The interaction of influence and autonomy combines to create an actor’s overall distinction in the network. The combination of influence and autonomy results in capacity, and the social reward for this capacity is distinction. An increase in one aspect can often, but not necessarily, lead to a decrease in the other. For example, a connection with someone with high accessibility can lead to a decrease in one’s autonomy, and new access comes with new constraints. However, the trade-off can maximize one’s overall distinction, so there are gains from the connection. Without such gains, the connection would not be formed.

\section{A Generalized Measure of Network Distinction}
Below, we present a measure of distinction centrality, determine the appropriate application of a scalar factor, and indicate the availability of an R program to facilitate easy implementation of the measure in further work.

\subsection{Distinction Centrality}

A formalization of this interaction between status and uniqueness allows for a measure of network distinction. Distinction as a theoretical concept is composed of status mediated by autonomy. A distinction measure can reflect this by considering an actor's status and then discounting their status by the status of their neighbors, independent of the actor. The more reliant an actor's neighbors are on the actor for their status, the lower the actor's ability to exert their influence, and thus the lower their distinction. In other words, autonomy is determined by how much one's neighbors rely on you to maintain their position in the network. 

Distinction centrality can be measured using the following expression:
\begin{equation}
\beta_i=Sx_i-{\frac{1}{n_{j=i}}\sum_{{j}\in{A},i\neq{j}}x_{j}} 
\end{equation}

Let $\beta_i$ be the distinction centrality of a given individual \textit{i}. The first term captures the status of a given actor. Let \textit{x} be the eigenvector centrality of the designated individual. Let S be a scalar that adjusts the social value placed on status relative to constraint, as discussed below. The second term captures the element of autonomy. Removing individual \textit{i} from the network, compute the eigenvectors of the remaining individuals and sum those that are connected to individual \textit{i}, taking the average eigenvector of neighbors \textit{j} as the constraint confronting node \textit{i} in the network. This is substantively the same mechanism as conducted in a jackknife resampling procedure, except that rather than forming a sampling distribution, we are measuring local constraints. Finally, subtract the status of the average neighbor connected to \textit{i} from \textit{i}’s scalar-adjusted eigenvector centrality in the whole network. 

When a node is an isolate, we set its eigenvector centrality to $x_i =0$. In the case of a network with no connections (such as an empty graph), some routines calculate it as $x_i =1.0$. However, given that it is still a network with no connections, setting isolates at $x_i =0$ is theoretically appropriate, considering that status is a reflection of connections. It would be inappropriate for an isolate to have the same status as the highest-status node in a large network. Additionally, given the second step of subtraction, it is only theoretically meaningful if set as $x_i =0$ in the case of distinction centrality.

\subsection{Scalar Adjustment}
First, a scalar is required because the relative value of status and autonomy is not inherently fixed in a social system. We might imagine situations where one desires more status at the expense of autonomy if the rewards to status are higher, or values autonomy more highly if the rewards to status are low. There cannot be an assumed unadjusted equivalence between the resulting eigenvectors of the two terms for this theoretical level.

Additionally, due to the hierarchical nature of networks, the average constraint in the network is greater than the average status. The high average constraint of their neighbors is often much larger than the impact of status. The result is that, at the level of the entire network, those who are deeply embedded among the most elite, without a basis of supporters, are very indistinct when considering the high constraints placed on them by their high-status alters. Thus, the calculated results without a scalar may fail to capture the appropriate value of status compared to the constraint.

While an arbitrary scalar is possible, it can yield erroneous results when applied to empirical networks. However, a scalar can be derived from the network itself due to the field positional interplay between status and constraint. Imagining two actors of equal status and personal constraint, one in an averagely more constrained system and the other in a less averagely constrained system, with average status otherwise equal. In such a case, the status of an actor in the more constrained system has a greater impact on their social distinction than in the less constrained system. With high constraint, status takes on greater importance in determining social distinction because it compels deference. In other words, it is the difference between a hierarchical system with and without constraints. Status with a weak constraint means a weaker compulsion of deference. The structure of constraint has a reverberative impact on social distinction beyond its contribution.

With this social understanding of how constraint interacts with the transmission of status, an endogenous scalar can be derived from the network. The average constraint of actors within the network divided by the average status of actors in the network creates a scalar that reflects the relationship between status and constraint. 

\[S=\frac{\frac{1}{n_{j=i}}\sum_{{j}\in{A},i\neq{j}}x_{j}}{\overline{x_i}} \]

Under conditions of high average constraint and low average status, the scalar is large, indicating the significant role of status in determining social distinction. Under conditions where the average constraint is low and the average status is high, the scalar is weaker, reflecting a system in which status is less important in determining social distinction. Under very large-scale conditions, the averages of status and constraint should trend towards 1, weakening the necessity of the scalar. However, under constrained conditions, particularly in elite networks with missing data on the social scaffolding that created the network elites, the scalar is more important for explaining action because the range of eigenvector network measures is otherwise artificially expanded due to the missing data.  

Without a scalar, networks with any degree of hierarchy see the average constraint overwhelm the average status because many low-status nodes are tied to high-status nodes. The advantage of this scalar is to ensure that the average scalar status in the network equals the average constraint in the network. Thus, total status is equal to total constraint, and from there it becomes a matter of distribution among members. From there, however, it enables comparability between networks that would otherwise be impossible. With the understanding that we have, through the scalar, adjusted for a network where one's combination of status and constraint is better or worse, networks can be compared so long as the unobserved social context within the network remains stable. 

\subsection{Distinction Centrality's Measurement Parameters}
In Figure 1, we present a variety of toy models that compare the effects of network interconnections on the status and constraint of actors. In each model, there are seven nodes, and node 1 is the best-connected node in the network. Thus, node 1 always has an eigenvector centrality of 1. However, the distinction of node 1 changes based on the interconnection of the other nodes, demonstrating how distinction centrality captures the question of constraint exerted by neighbors. 

\textit{Model (a)} is a star graph where node 1 is completely connected to all other nodes, yet the other nodes share no connections. Under such conditions, node 1's eigenvector centrality is 1.00. Yet, node 1's constraint is 0.00 because, when node 1 is removed from the network, all of node 1's neighbors have no connections and thus an eigenvector centrality of 0.00. Table 1 displays the various measurements for each node. The hub of a star graph thus has the maximum possible distance between status and constraint. However, with additional spokes added to a star graph, the scalar adjusted distinction measure increases. The hub of a star graph with seven spokes is more distinct than a hub with six, allowing a degree of comparability across networks. A network composed of a dyad would have a scalar adjusted distinction of 0.00 because the average constraint is 0. Distinction becomes a meaningful phenomenon only in networks of three or more nodes. 

\textit{Model (b)} presents an idealized depiction of a structural fold \citep{vedres2010structural}. The resulting measurement scores are in Table 2. While node 1 maintains high eigenvector centrality, the eigenvector centrality of the other nodes increases, while their constraint decreases. It is actually the constraint of Node 1 that increases, reaching 1.00 because all its neighbors are maximally interconnected in their respective subgroups. This position of brokerage is more constrained than the star graph because the other actors are connected in subgroups. The actors in the subgroup are able to combine their interests to mitigate against the independence of Node 1 in the center. In an unadjusted measure of distinction, the different nodes become remarkably close in their distinction measure. With a scalar adjustment, the stratification increases, giving Node 1 a bit more privilege, but also demonstrating the extent to which Node 1's distinction is eroded. In a maximally connected graph, the distinction of each node would also go to 0.00, as there is no network differentiation among nodes. 

\begin{figure}
    \captionsetup[subfigure]{font=footnotesize,labelfont=footnotesize}
    \centering
     \begin{subfigure}[b]{0.35\textwidth}
        \includegraphics[width=1.0\textwidth]{Plots/star.png}
            \caption{Star Graph}
            \label{fig:star}
    \end{subfigure}
     \begin{subfigure}[b]{0.35\textwidth}
        \includegraphics[width=1.0\textwidth]{Plots/sf.png}
            \caption{Structural Fold}
            \label{fig:sf}
    \end{subfigure}
    \begin{subfigure}[b]{0.35\textwidth}
        \includegraphics[width=1.0\textwidth]{Plots/intra.png}
            \caption{2-Connected Subgraphs}
            \label{fig:intra}
    \end{subfigure}
    \begin{subfigure}[b]{0.35\textwidth}
        \includegraphics[width=1.0\textwidth]{Plots/inter.png}
            \caption{2-Connected Subgraphs with Long Tie}
            \label{fig:inter}
    \end{subfigure}
    \begin{subfigure}[b]{0.35\textwidth}
        \includegraphics[width=1.0\textwidth]{Plots/circle.png}
            \caption{Circle Graph}
            \label{fig:inter}
    \end{subfigure}
    \begin{subfigure}[b]{0.35\textwidth}
        \includegraphics[width=1.0\textwidth]{Plots/circle-plus-one.png}
            \caption{Circle Graph with Pendant Node}
            \label{fig:inter}
    \end{subfigure}
    \caption{Distinction Centrality in toy graphs. Nodes with the highest distinction are shown in red, nodes with positive distinction are shown in blue, nodes with negative distinction are shown in tan, and nodes with distinction close to zero are shown in purple.}
    \label{fig:toys}
\end{figure}

\begin{table}
\centering
\caption{\label{tab:star}Distinction centrality scores for the star graph.}
\centering
\begin{tabular}[t]{lrrrr}
\toprule
Nodes & $\beta_i$ & $s_i$ & $\kappa_i$ & $\alpha$\\
\midrule
1 & 1.230 & 0.707 & 0.000 & 1.739\\
2, 3, 4, 5, 6, 7 & -0.205 & 0.289 & 0.707 & 1.739\\
\bottomrule
\end{tabular}
\end{table}

\begin{table}
\centering
\caption{\label{tab:sf}Distinction centrality scores for the structural fold graph.}
\centering
\begin{tabular}[t]{lrrrr}
\toprule
Nodes & $\beta_i$ & $s_i$ & $\kappa_i$ & $\alpha$\\
\midrule
1 & 0.026 & 1.000 & 1.000 & 1.026\\
2, 3, 4, 5, 6, 7 & -0.004 & 0.608 & 0.628 & 1.026\\
\bottomrule
\end{tabular}
\end{table}

\begin{table}
\centering
\caption{\label{tab:tab:intra}}
\centering
\begin{tabular}[t]{rrrrrr}
\toprule
Node & Distinction & SDistinction & Status & Constraint & Scalar\\
\midrule
1 & 0.195 & 0.369 & 1.000 & 0.805 & 1.174\\
2 & -0.257 & -0.170 & 0.500 & 0.757 & 1.174\\
3 & -0.257 & -0.170 & 0.500 & 0.757 & 1.174\\
4 & 0.048 & 0.155 & 0.618 & 0.570 & 1.174\\
5 & 0.048 & 0.155 & 0.618 & 0.570 & 1.174\\
6 & -0.257 & -0.170 & 0.500 & 0.757 & 1.174\\
7 & -0.257 & -0.170 & 0.500 & 0.757 & 1.174\\
\bottomrule
\end{tabular}
\end{table}

\begin{table}
\centering
\caption{\label{tab:tab:inter}}
\centering
\begin{tabular}[t]{rrrrrr}
\toprule
Node & Distinction & SDistinction & Status & Constraint & Scalar\\
\midrule
1 & 0.251 & 0.361 & 1.000 & 0.749 & 1.11\\
2, 7 & -0.047 & 0.025 & 0.654 & 0.701 & 1.11\\
3, 6 & -0.299 & -0.248 & 0.465 & 0.765 & 1.11\\
4, 5 & -0.025 & 0.042 & 0.612 & 0.637 & 1.11\\
\bottomrule
\end{tabular}
\end{table}

\begin{table}
\centering
\caption{\label{tab:tab:circle}}
\centering
\begin{tabular}[t]{rrrrrr}
\toprule
Node & Distinction & SDistinction & Status & Constraint & Scalar\\
\midrule
1 & 0.555 & 0 & 1 & 0.445 & 0.445\\
2 & 0.555 & 0 & 1 & 0.445 & 0.445\\
3 & 0.555 & 0 & 1 & 0.445 & 0.445\\
4 & 0.555 & 0 & 1 & 0.445 & 0.445\\
5 & 0.555 & 0 & 1 & 0.445 & 0.445\\
6 & 0.555 & 0 & 1 & 0.445 & 0.445\\
7 & 0.555 & 0 & 1 & 0.445 & 0.445\\
\bottomrule
\end{tabular}
\end{table}



\textit{Model (c)} presents a graph that is 2-connected in the absence of node 1. Each subgraph is one connection short of forming complete subgraphs in the absence of Node 1. Compared to the structural fold, connections 3-2 and 6-7 have been removed. Measurement results are in Table 3. Of the demonstrative networks here, this is the one that most approximates a three-tiered ordered hierarchy. The resulting scalar adjusted distinction measures well reflect this stratification, with Node 1 retaining its distinction, Nodes 4 and 5 failing in the middle position, and Nodes 2, 3, 6, and 7 falling to the bottom. 

\textit{Model (d)} is the same 2-connected graph with the addition of a long tie connecting nodes 2 and 7. This long tie stratifies the network even more than in Model (c), bringing Nodes 2 and 7 quite close to 4 and 5. The formation of the long tie served to bolster the individual distinction of 2 and 7 more than the formation of the tie that would create the completely connected subgraphs as in Model (b). In fact, all nodes except 3 and 6 have higher distinction in such a case. This reflects the positional idea that distinction is a matter of stratification, and cohesion is at odds with creating stratification. While the status of Nodes 2 and 7 actually increases beyond that of Nodes 4 and 5 due to the long tie, the constraint also 
increases as they are now both further constrained by the influence of Node 1, their mutual neighbor.   

\subsection{Distinction with Structural Holes vs. Structural Folds}

%Clique addition experiment

%Add cliques to star graph and see what happens

%See what happens with the caveman

%Placehold graphs
%\begin{figure}
%   \captionsetup[subfigure]{font=footnotesize,labelfont=footnotesize}
%    \centering
%     \begin{subfigure}[b]{0.45\textwidth}
%        \includegraphics[width=1.0\textwidth]{Plots/sf.png}
%            \caption{Star Graph}
%            \label{fig:Placehold}
%    \end{subfigure}
%     \begin{subfigure}[b]{0.45\textwidth}
%        \includegraphics[width=1.0\textwidth]{Plots/sf.png}
%            \caption{Structural Fold}
%            \label{fig:Placehold}
%    \end{subfigure}
%    \caption{Structural Holes with more added subgraphs.}
%    \label{fig:toys}
%\end{figure}

\subsection{Long Ties and Hierarchy}
This section highlights how the hierarchy is really formed out of the redundant connections of the subgraphs. While a star graph clearly creates a hierarchy, when additional weight is added to the broker's alters, the hierarchy can be flipped, especially when they form denser subgraphs. This is in keeping with the observed structural holes above. 

Graph (b) is the same as graph (a),but with Nodes 12, 13, and 14 combined into a clique with Node 9. Graph (c) is the same as graph (b) with the addition of Nodes 21, 22, and 23 as personal connections of Nodes 12, 13, and 14. Finally, graph (d) is the same as graph (c) with the addition of three long ties across subgraphs connecting only Nodes 17 with 23 and 11 with 18.

As Figure 5 graph (d) demonstrates, when those subgroups become connected redundantly not through the central figure, but through the sparse interconnection of the other members, then Node 9's position of distinction becomes highly elevated.

Thus, there is a U shaped pattern where at low levels of interconnection between sparse subgroups, mediated by a central node, the distinction level of the broker is high. However, as those subgraphs become more densely connected within themselves, the position of the central node is weakened as status accrues more heavily in those local subgraphs. However, when sparse long connections then form to connect the subgraphs, Node 9 achieves a high level of distinction again because the overall constraint in the network increases with the added ties between the subgraphs. The resulting decrease in the scalar works to stratify the distinction distance between Node 9 and other high status competitors, while simultaneously decreasing the constraint forced upon Node 9 

\begin{figure}
    \captionsetup[subfigure]{font=footnotesize,labelfont=footnotesize}
    \centering
     \begin{subfigure}[b]{0.45\textwidth}
        \includegraphics[width=1.0\textwidth]{Plots/political1.png}
            \caption{High Subgraph Density, Low Graph Density}
            \label{fig:Placehold}
    \end{subfigure}
     \begin{subfigure}[b]{0.45\textwidth}
        \includegraphics[width=1.0\textwidth]{Plots/political2.png}
            \caption{Broker Given Independent Foundations}
            \label{fig:Placehold}
    \end{subfigure}
         \begin{subfigure}[b]{0.45\textwidth}
        \includegraphics[width=1.0\textwidth]{Plots/political3.png}
            \caption{Increased Independent Foundations}
            \label{fig:Placehold}
    \end{subfigure}
     \begin{subfigure}[b]{0.45\textwidth}
        \includegraphics[width=1.0\textwidth]{Plots/political4.png}
            \caption{All Subgraphs Connected by Long Ties}
            \label{fig:Placehold}
    \end{subfigure}
    \caption{Increasingly connected graphs with central broker.}
    \label{fig:Evolving Political Network}
\end{figure}


\begin{table}
\centering
\caption{\label{tab:tab:search}}
\centering
\begin{tabular}[t]{rrrrrr}
\toprule
Node & SDistinction(a) & SDistinction(b) & SDistinction(c) & SDistinction(d)\\
\midrule
5 & 0.6585 & 0.4872 & 0.4744 & 0.3072\\
6 & 0.3232 & 0.2903 & 0.2836 & 0.1511\\
7 & 0.5717 & 0.4535 & 0.4536 & 0.3182\\
8 & 0.0504 & 0.1521 & 0.3420 & 0.1838\\
9 & 0.1936 & -0.1752 & 0.1974 & 0.5377\\
10 & 0.1937 & 0.2152 & 0.2223 & 0.1130\\
\bottomrule
\end{tabular}
\end{table}


\begin{table}
\centering
\caption{\label{tab:tab:search}}
\centering
\begin{tabular}[t]{rrrrrr}
\toprule
Node & Status(a) & Status(b) & Status(c) & Status(d) \\
\midrule
5 & 1.0000 & 1.0000 & 0.9683 & 0.9638\\
6 & 0.8957 & 0.8566 & 0.7813 & 0.7776\\
7 & 0.9854 & 0.9401 & 0.8468 & 0.8513\\
8 & 0.7924 & 0.7636 & 0.9056 & 0.8993\\
9 & 0.5239 & 0.7284 & 1.0000 & 1.0000\\
10 & 0.8277 & 0.7897 & 0.6775 & 0.6838\\
\bottomrule
\end{tabular}
\end{table}



\subsection{Implementation}
To conduct the analyses below, we use a custom function written for the \textit{R} statistical computing environment \citep{R} software. The function code is reproduced in the \hyperref[sec:appendix]{the Appendix}.

\section{Betweenness vs Distinction: Revisiting the Medici Network}
The interaction of influence and autonomy combines to create an actor’s overall distinction in the network. The combination of influence and autonomy results in capacity, and the social reward for this capacity is distinction. An increase in one aspect can often, but not necessarily, lead to a decrease in the other. For example, a connection with someone with high accessibility can lead to a decrease in one’s autonomy, and new access comes with new constraints. However, the trade-off can maximize one’s overall distinction, so there are gains from the connection. Without such gains, the connection would not be formed.

One of the consecrated papers of social network analysis is Padgett and Ansell’s (1993) analysis of the centrality of the Medici to Florentine elite marriage and business networks in the 15th century. The empirics led to the theory of robust action, demonstrating that one’s position in the network can become the basis of one’s interests, and that through balancing competitors, an actor can amplify their social position. 

A puzzle that results from the empirics however is how the balance between antagonistic factions created by the Medici came to be undone within a decade of Padgett and Ansell’s focus. Brokerage created social stability, yet in 1432 Rinaldo Albizzi as leader of the combination of conservative old oligarchic families seized the reigns of the Florentine government, exiling the leading Medici from Florence. Why was the network position of the Medici as brokers so unstable? In other words, what were the network incentives of the old conservative families to attempt to remove the Medici from their network position? 

The opportunities for an increase in social distinction explain the attempt by the conservative oligarchs to remove the Medici from the network. Padgett and Ansell built their network of Florentine marriages and economic exchange as a series of blockmodels, which capture the inter-relations among families \citep{padgett1993robust} 1276. Underneath the blockmodel are even more families in their Appendix B, such that there were 92 elite families overall considered (1276). The politics of Florence centered around family alliances consecrated through marriage, trade, and friendship ties (). However, over the years many families had developed cadet branches, with many old families having 10 or more households, with the Strozzi and Bardi having more than 50. Newer families had smaller, less ancient family networks, and so inter-family political cooperation was important for their social protection (150-1). Family capacity through size and wealth was then channeled into inter-elite marriages, creating the networks of power and influence through which Florentine politics was channeled. The unity of the households within a single lineage was variable however, often depending on their own marriages. For example, the Bardi family largely sided with the old families against the Medici, but the marriage of Contessina de Bardi to Cosimo de Medici (129) meant that the family was not united in opposing the Medici. 
	
To empirically examine the opportunity for gaining social distinction in the Florentine network, Figure 2a from Padgett and Ansell was reproduced as an edgelist, with each tie between blocks counted as a single connection, regardless of its quality or frequency in the original Figure 2a. It is essentially a simplified network of blocks, treated as network actors, that are socially connected, indicated by the family names originally used by Padgett and Ansell. Ignoring the intensity of ties is unlikely to change the results given that most are singly connected in the original figure, and the multiple ties are still fragmented among a variety of households within a family, so they could inordinately strengthen connections rather than explain fracturing that occurred within families. Examining the Florentine network with the above measure of social distinction explains the transformation from political stability to fragmentation as a result of actor level network incentives.

Results are presented in Table X. In the original state of the network, the Medici had the highest betweenness (), while the Peruzzi had the highest status. The Albizzi, with the second highest status, owed their high status in part to their connection with the Medici. Without the marriage connection to the Medici, the Albizzi would still have high status in the networks, but would otherwise be eclipsed by the Guasconi, who were also connected to the Medici and other major families. The Medici were actually of relatively middling status in the network, clearly deriving their power from the brokerage leadership.         


\begin{figure}
    \captionsetup[subfigure]{font=footnotesize,labelfont=footnotesize}
    \centering
     \begin{subfigure}[b]{1.0\textwidth}
        \includegraphics[width=1.0\textwidth]{Plots/medici.png}
            \caption{Medici Network}
            \label{fig:medici}
    \end{subfigure} \\
     \begin{subfigure}[b]{1.0\textwidth}
        \includegraphics[width=1.0\textwidth]{Plots/nomedici.png}
            \caption{Medici Network without the Medici}
            \label{fig:nomedici}
    \end{subfigure}
    \caption{}
    \label{fig:medici}
\end{figure}

\begin{table}
\centering
\caption{\label{tab:label}}
\centering
\begin{tabular}[t]{lrrrrr}
\toprule
Node & Distinction & SDistinction & Status & Constraint & Scalar \\
\midrule
Medici & -0.040 & 0.672 & 0.576 & 0.615 & 2.237\\
Guicciardini & -0.046 & 0.236 & 0.228 & 0.275 & 2.237\\
Ginori & -0.697 & -0.515 & 0.147 & 0.844 & 2.237\\
Tornabuoni & -0.158 & 0.054 & 0.172 & 0.330 & 2.237\\
Albizzi & 0.330 & 1.496 & 0.943 & 0.613 & 2.237\\
Guasconi & 0.167 & 1.257 & 0.882 & 0.715 & 2.237\\
Cocco-Donati & -0.628 & -0.456 & 0.139 & 0.768 & 2.237\\
Valori & -0.100 & -0.060 & 0.032 & 0.131 & 2.237\\
Dietisalvi & -0.105 & -0.063 & 0.034 & 0.139 & 2.237\\
Dal'Antella & -0.105 & -0.063 & 0.034 & 0.139 & 2.237\\
Davanzati & -1.000 & -1.000 & 0.000 & 1.000 & 2.237\\
Orlandini & -1.000 & -1.000 & 0.000 & 1.000 & 2.237\\
Guadagni & -1.000 & -1.000 & 0.000 & 1.000 & 2.237\\
Fioravanti & -1.000 & -1.000 & 0.000 & 1.000 & 2.237\\
Bishceri & -1.000 & -1.000 & 0.000 & 1.000 & 2.237\\
Ardinghello & -0.153 & -0.092 & 0.049 & 0.202 & 2.237\\
Da Uzzano & -0.705 & -0.441 & 0.213 & 0.918 & 2.237\\
Benizzi & -0.073 & 0.310 & 0.310 & 0.383 & 2.237\\
Baroncelli & -0.103 & 0.034 & 0.111 & 0.214 & 2.237\\
Velluti & -0.391 & -0.075 & 0.255 & 0.646 & 2.237\\
Lamberteschi & -0.182 & -0.109 & 0.058 & 0.240 & 2.237\\
Solosmei & -0.165 & 0.049 & 0.174 & 0.339 & 2.237\\
Della Casa & 0.002 & 0.727 & 0.586 & 0.583 & 2.237\\
Altoviti & -0.680 & -0.413 & 0.216 & 0.896 & 2.237\\
Peruzzi & 0.327 & 1.564 & 1.000 & 0.673 & 2.237\\
Panciatichi & -0.251 & 0.491 & 0.600 & 0.851 & 2.237\\
Strozzi & 0.008 & 0.844 & 0.675 & 0.667 & 2.237\\
Aldobrandini & -0.771 & -0.487 & 0.229 & 1.000 & 2.237\\
Castellani & -0.257 & 0.297 & 0.448 & 0.705 & 2.237\\
Rucellai & -0.486 & -0.295 & 0.155 & 0.641 & 2.237\\
Rondinelli & -0.121 & 0.422 & 0.439 & 0.560 & 2.237\\
Pepi & -0.319 & -0.192 & 0.103 & 0.422 & 2.237\\
Scambrilla & -0.319 & -0.192 & 0.103 & 0.422 & 2.237\\
\bottomrule
\end{tabular}
\end{table}



In a comparison of the network with the Medici compared to the one with them absent, one notices that the status of the oligarchic families does not automatically increase. While those further away from the Medici such as the Strozzi and Panciatichi would see their status increase in the Medici’s absence as the center of the network would shift its center towards them, the Guasconi and the Albizzi would lose by such a straightforward measure of changing status. Considering the matter of constraint, The Guasconi and Albizzi would actually also become more constrained in the absence of the Medici from the network. Thus, in a situation where their status decreases and their constraint increase, these families still chose to act against the Medici! 

The difference in status and distinction for the Guasconi and Albizzi would still be lower without the Medici in the network than with the Medici. However, when adjusting for the level of status and constraint in the system, the Guasconi and Albizzi would, along with the rest of the conservative oligarchs, increase their social distinction. The overall decrease within the system as a whole of average status and an increase in constraint would make for the increasing importance of the status that the Guasconi and Albizzi did have at the level of the system. The inequality within the system would essentially increase without the presence of the Medici, and ending up in the elite cluster of a more unequal network of elites was an increase in social distinction sufficient to entice members of the Guasconi and Albizzi blocks. In contrast to the application of an edge cutting algorithm that may shear the Guasconi and Albizzi blocks from the Medici due to the weight of their connections with the rest of the network, the members of the families in these blocks had distinction to be gained, as long as their own personal connections did not otherwise tie them to the Medici.  

It was not just Cosimo de Medici who derived his sphinxlike qualities from his network position, but Rinaldo degli Albizzi was likewise noted for his unwillingness to take strong stances on behalf of the conservative faction of old elite families (Kent 1978:289). Given Rinaldo’s similarity to Cosimo’s brokerage network position, the structural incentives towards equivocation were similarly expressed. However, Rinaldo’s structural opportunities were transformed with the death of Niccolò da Uzzano, the previous leader of the conservative faction. While the Uzzano’s are attached to the Guasconi block in the original graph of relations, the actually shared deep family relations with the Bardi who fell into the same block as the Guasconi. Per Table B1 in Padgett and Ansell (1314-6), the Bardi were third in gross family wealth after the Strozzi and Medici. Niccolò’s mother was a Bardi, and his brother married into the family (Kent 1978:166), elevating Niccolò’s authority beyond what might otherwise be discernable from the original graph. The Uzzano and Bardi leadership of the conservatives balanced the old elites away from Rinaldo, making ties to the Medici more important for their network position. Niccolò was also seen largely as moderate despite leading the old families against the new families (Kent 1978:256). However, Niccolò’s death in 1432 created the gap in the conservative faction’s leadership for Rinaldo to fill. The center of power within the network of old families was renegotiable, and in exchange for eschewing the Medici, the Rinaldo degli Albizzi could seize leadership and shift the networks emphasis towards himself. The secondhand data from Padgett and Ansell’s block models does not allow for a clear demonstration of the movement, but the historiography would indicate the underlying shifts in network opportunity.      

With Bernardo Guadagni’s rise to Gondolfier, Albizzi and the old families made their move to exile Cosimo and Lorenzo de Medici, seemingly to seize the levers of city government through factional decapitation, thereby avoiding destructive conflict (289-90). While the conservatives seized government, the Medicean’s were not undone by the loss of their leaders but rather resisted without prompting crisis (299). The moment of crisis came with the lottery, whereby chance the Medici supporters came to dominate the Priors (328). The resulting re-entry of the Medicean’s to government power threatened the conservative faction, and Rinaldo’s response was to mobilize the forces of the conservative families to seize the castle where the priors were meeting. The Medicean’s gained word and were able to reinforce the castle before the conservatives could arrive. In this moment of constitutional crisis, Palla Strozzi refused to commit his large force of support to Rinaldo’s cause, and even the Peruzzi demobilized upon receipt of a deal from the Medicean’s (331-4). The crisis itself was negotiated down through the intervention of the Pope, but the crisis shifted momentum to the Medicean’s side through the fragmentation of the conservatives and the outpouring of armed support flowing in from outside the city walls (Kent 1978:330-4). The leadership of the old families were subsequently undone for their crimes (Kent 1978:335-48). The conservatives would clearly improve their distinction within Florence with the removal of the Medici, yet in the end the elite network does not explain their defeat. The threatened major social losses to the newer families explain their dogged resistance to the conservative ascension, yet the support for the Medici’s with the Pope and beyond Florence was critical as the balance of force among the elite families was in the conservative’s favor (see Padgett and Ansell Appendix B).  

\section{Group Detection vs Distinction: Revisiting the Karate Club}
Another consecrated paper in network analysis is Zachary’s (1977) examination of the fissure of a university karate club network. Zachary found that the tension in the group centered around questions of compensation for the outside karate teacher with the internal student group leadership. However, Zachary finds that the dispute was channeled and resolved along an already existing division in the underlying friendship connections among the club members (1977:454-5).   

The paper was an early application of an algorithm to predict the lines along which the group would divide. The original Ford-Fulkerson Labeling algorithm was a network flow model that sought to use the length between nodes to detect community membership (1977:454). However, in the original algorithm, Node 9 was misclassified to the Officer’s network, whereas empirically Node 9 joined the network of the karate teacher, Mr. Hi. The original discrepancy was explained in reference to Node 9 being near to achieving a black belt, and so the institutional opportunity costs overrode the social network costs. 

Contrary to hopes, the application of the distinction centrality measure was not able to explain why Node 9 left for Mr. Hi's network. The original hypothesis was that, because of the Officer's network closely resembles a star graph, that the level of distinction that Node 9 would command in the network would be low compared to the more heterogeneous network of Mr. Hi. Results are displayed in Table X. The measure demonstrates that Node 9 would have had an overall greater relative distinction if he had stayed in the Officer's network. The idea being that, in Mr. Hi's network, Node 9 would be more constrained than in the officer's network, but would have higher status because he was directly connected to the high status center in an otherwise stratified network. 

The exercise, however, focuses attention on the boundaries of the field. The original explanation for the discrepancy points to Node 9's large karate capital. If the field is bounded as the ``karate club,'' Node 9's decision is unexpected, but if the field is ``karate practitioners'' then there might be much of the field that is going unobserved in Zachary's original study. We do know that some of the students, including Node 9, practiced at Mr. Hi's dojo as well as the club. The fact that Node 9 had club connections only to the most central members of both the officer's and Mr. Hi's networks would indicate that Node 9 only affiliated with likely similarly advanced members of the club. If Node 9 was similarly affiliating with advanced students who only attended the dojo, then this would likely influence the decision regarding distinction at the field level. Failure to completely capture the positions in the appropriately defined field can result in measures that fail to accurately capture empirical outcomes. 

\section{Distinction as Search Algorithm}

Given the value that eigenvector-style centrality measures have had in underpinning the logic of internet search algorithms \citep{kleinberg99, brin_page98, langville2005survey-572}, the application of distinction centrality in the case of a reference network is briefly considered. As the social logic of distinction centrality involves jointly considering the interaction of status and constraint, so in a reference network, the distinction measure will give greater priority to the uniqueness of a node's flow of affiliations. 

A toy reference network is presented in Figure 3. The network was designed to contrast status versus distinction within a graph, featuring a clear cluster of nodes with an elevated level of density and tenuously connected to a node with a high degree of popularity external to the main, denser cluster. Such a graph is supposed to represent a contrast between an orthodox set of mutual endorsements and the emergence of a heterodox set of connections.

The resulting distinction measures are presented in Table 6. The results are ranked in order of highest to lowest scaled distinction centrality. Node 8 measures highest in both scaled distinction and status, but immediately after Node 9 and Node 6 switch places in their ordering. Node 6 has the second highest status, but the third highest scaled distinction, while Node 9 goes from third in status to second in scaled distinction. The measure rewards the independent connections Node 9 has with Nodes 13 and 14 over the heavier constraints imposed on Node 6. \footnote{In the absence of Node 8's and its constraint effect on Node 6, Node 6 would become the most distinct and highest status.} Node 10's relative position in the social hierarchy is consistent at fourth in both the status and scaled distinction measures.

\begin{figure}
    \captionsetup[subfigure]{font=footnotesize,labelfont=footnotesize}
    \centering
     \begin{subfigure}[b]{0.45\textwidth}
        \includegraphics[width=1.0\textwidth]{Plots/search.png}
            \caption{Reference Network Example}
            \label{fig:search}
    \end{subfigure}
    \caption{Example of a non-directed network of references, e.g. citation network, internet links, etc.}
    \label{fig:search}
\end{figure}

\begin{table}
\centering
\caption{\label{tab:tab:search}}
\centering
\begin{tabular}[t]{rrrrrr}
\toprule
Node & Distinction & SDistinction & Status & Constraint & Scalar\\
\midrule
8 & 0.4939 & 1.0039 & 1.0000 & 0.5061 & 1.5099\\
9 & 0.3265 & 0.7680 & 0.8657 & 0.5392 & 1.5099\\
6 & 0.1530 & 0.6442 & 0.9633 & 0.8103 & 1.5099\\
10 & 0.1216 & 0.5367 & 0.8141 & 0.6925 & 1.5099\\
5 & 0.1837 & 0.3675 & 0.3603 & 0.1766 & 1.5099\\
7 & -0.0898 & 0.1983 & 0.5649 & 0.6547 & 1.5099\\
1, 2, 3, 4 & -0.2298 & -0.1807 & 0.0963 & 0.3260 & 1.5099\\
16 & -0.3764 & -0.2994 & 0.1510 & 0.5273 & 1.5099\\
15 & -0.5459 & -0.4349 & 0.2175 & 0.7634 & 1.5099\\
13, 14 & -0.5780 & -0.4601 & 0.2313 & 0.8094 & 1.5099\\
11, 12 & -0.707 & -0.5707 & 0.2672 & 0.9742 & 1.5099\\
\bottomrule
\end{tabular}
\end{table}

The heterodoxy of Node 5 is elevated into the fifth position of the scaled distinction hierarchy compared to the sixth position it has in the status hierarchy. Node 7 correspondingly falls into the sixth position of the scaled distinction hierarchy from the fifth position in terms of status. However, scaled distinction centrality also elevates Nodes 1, 2, 3, and 4 which buttress 5's popularity. These nodes have the lowest status, yet leap over the hanger's on to the core members of the main orthodox cluster. They are greatly penalized for their high level of constraint, whereas Node's 1, 2, 3, and 4 are empowered by 5's own lower status position. If the goal is to elevate uniqueness, this makes sense as Nodes 11 through 16 are likely to conform to their higher status, and high distinction, connections. Nodes 7, 8, 9, and 10 already implicitly represent their contributions. Additionally, the application of the scalar is important in the final results. Without the scalar, Node 5 would actually be the third ranked node in terms of distinction, leaping Node 6. Such a result may be desirable in technical applications where uniqueness should be privileged over social hierarchy. On the whole, the scaled distinction centrality measure thus works to elevate the uniqueness of a node in search applications.   

\section{Implications for Theories of Distinction}
\subsection{Long Ties and Opportunities for Distinction}
Given that networks are not static, but evolving, the incentives to maximize distinction within a network can inform the evolution of culture as a function of network reorganization. As cultural capital is an outcome of both status and uniqueness, the same calculus of optimizing network position can be extended.

Hearkening to the stereotypical model of a village society, we can imagine a social world composed of dense local clusters interconnected only through elite networks. In such a network, most individuals would have dense ties within their local subgroup, but have few if any ties outside of the subgroup. Such extra-local relationships would be managed by a handful of elites who would derive their status from the deference paid to them locally and through their brokerage with the wider world (see Tilly 1978 on the brokerage role of French priests and the nobility in France prior to the Revolution). The incentives for distinguishing oneself in the context of a dense local subgroup through local stratification was liable to engender a decline in the actors status. In a tightly connected subgroup, stratification is low, and so the ability to differentiate is also low. Any initial actions are likely to lead to a loss of status greater that outweighs the gain in autonomy. 

With these initial network conditions, the network pressures for cultural innovation are low. However, as economic and technological change brings actors into greater contact, the network of positions also undergoes transformation. The changes also begin to create greater local stratification. These changes create an opportunity structure for actors to create new connections in the network to increase their own distinction. 

We would expect to see in large datasets a tendency for people who are locally constrained and of middling status to attempt to reduce their constraint by attempting to form long ties because it can improve both their status and their autonomy. Those of high status might not jeopardize their distinction by associating with those of lower standing, while those of low status are constrained by the reluctance of others to associate with them. The expected result would be a flourishing of long tie formation in the middle levels of the distinction hierarchy. This prediction can be tested against large longitudinal, well-circumscribed networks such as \citep{bearman2004chains} in the future. 

The insight that it is the middling levels that are most likely to form long ties to create their distinction indicates that it is structures with a level of stratification and the opportunity to form long ties that are most likely to create cultural heterogeneity, or in other words a flourishing of subcultures. Colleges are not merely opportunities for people from many backgrounds to recombine their cultural repertoires, but are actually network structures that encourage the pursuit of cultural distinction given the long ties that accompany the students new position. The expansion of higher education beyond the old elites in the 1950s and 60s was a network transformation that beget the major cultural shifts that began in these institutions. These cultural transformations were neither inherent in the students nor the institutions, but came out of the network structure that enabled the pursuit of cultural distinction to match the newly acquired network position.  

The cultural practices of elites in a society are similarly dictated by the tradeoffs of status and autonomy that derive from network structure. The French elite of Bourdieu presents the ideal haute elite in terms of a stratified cultural disposition from the lower status members of society. However, the modern American elite has been characterized not by its sanctification of a limited set of cultural practices, but rather by its embrace of variety, or its cultural omnivorousness \citep{bryson1996anything}. Sentence that connects to this citation-> \citep{lizardo2014omnivorousness}. The divergence in an elite interested in limited consecration and an elite interested in omnivorousness can be found in the network structure created by national geography and institutions. France, dominated by a single major capital as far back as the 1600s, had a singularly concentrated national elite who formed dense interconnections. The opportunity for enterprising distinction was foreclosed by interconnection. America's varied national elite geography conversely created locally cohesive elites that were connected less intensely at the national level. When brought into deeper integration after 1945, the elite themselves similarly formed long ties, creating the positional opportunity for the proliferation of elite cultural forms.      

While the above exploration of how network shape changes diversity in cultural distinction practices is at least informative about how understanding the network of positions structures the cultural markers of distinction. 

%College as inherently distinction based institution makes liberals a thing, rather than if you only draw from caves

\subsection{Capital Conversion as Network Integration of Fields}
One of the problems with the relationship between multiple fields is how capital might be converted between fields. The ``rate'' between fields has always been a matter of confusion because it is difficult to quantify. Yet, it is an essential feature of an individual's navigation between the positional hierarchies of various fields. Seeking to convert a distinct position from one field to a better position in another is a critical strategy in the overall positional portfolio of agents.

However, when thought of as horizontally interlinked networks of vertical fields, the problem of capital conversion becomes tractable as a network problem. If someone from an economic field (broadly speaking) were to associate with someone high in some form of cultural capital, it refracts into the economic network as well because associations are at the level of the connections between individuals. Personal associations may be somewhat field-specific decisions, but unless field boundaries are rigorously enforced, associations between individuals are largely portfolio-level decisions. 

Such a network position perspective aligns with the observation that those high in different types of capital are often willing to associate with one another, borrowing their mutual endorsements to advance their portfolios. Thus, those who are ``low'' in certain forms of capital can associate with those who are ``high'' in another type, insofar as it does not create undesirable network ties. The prototypical example is the artist and the benefactor. While the artist imparts cultural capital to the benefactor, the economic rewards, even if meager, are necessary for the artist. However, what is important for the benefactor is that the artist has the \textit{right type} of low economic capital. If the economic capital of those at the height of the economy is built upon a workforce, associations with those who buttress their economic position would erode the overall portfolio. Such associations would close the distance that is necessary for having a distinct position. 

Instead, one can associate with those who are low in economic capital \textit{insofar} as they are not interconnected with the economic structures that buttress the position of the elites of the economic field. The artist connected to, or a part of, the proper working class undermines the structure of the economic field. The artist with a university education who may work a menial job is, through their social distance from the proper working class, does not erode the structure of the field. The conversion of cultural capital into economic capital is fundamentally about not having connections to the core structures of the economic field. Thus, those low in one type of capital and high in another will have the most success if the capital they lack is largely peripheral to the main structure of stratification in the field.

Note: "the real logic of the functioning of capital [is the] conversion from one type to another" (1997,54) in Scott 238.

\begin{figure}
    \captionsetup[subfigure]{font=footnotesize,labelfont=footnotesize}
    \centering
     \begin{subfigure}[b]{1.0\textwidth}
        \includegraphics[width=1.0\textwidth]{Plots/PortfolioIntegration.png}
            \caption{Model of Viable Portfolio Integration}
            \label{fig:medici}
    \end{subfigure} \\
    \caption{}
    \label{fig:integration}
\end{figure}

Similarly, those who are in distinct positions created by commonality in the individuals that structure the lower rungs of the network structure are able to easily transfer their position across fields. A sports star of low background being elevated in their social status by the working classes is no threat to the economic elites to affiliate freely with such an individual. The fields may differ, but they share a common scaffolding of working individuals.

Thus, the demarcation of the field becomes incredibly important in determining whether capital can be transferred. Within a field, practices become stratified, but when the field becomes defined as separate, capital among those most distinct in their fields can be exchanged if they share a common scaffolding. The key is the stratification rather than the cultural practice. Sports might be declasse from the perspective of cultural stratification, but the stratification is generally equivalent (general equivalence citation). Furthermore, stratifying the consumption experience is another way to make the ``same'' cultural object maintain the boundaries of distinction. 

Floor or box seats, or season tickets, reinforce distinction while allowing for common consumption. Similarly, pop culture can be popular because of this aspect of stratification, the pedestal on which a producer of culture is placed. Common consumption is not equivalent to common performance. Common consumption has the elevation of the producer, but common production lacks stratification, and thus must be stratified through avoidance by the distinct. Stratified common consumption is acceptable to elites, but un-stratified common consumption is to be avoided. However, those whose economic position may more closely match those who build up the scaffold must avoid common consumption because it would push their position into being coincident with those whom they are seeking to differentiate themselves from. This explains the avoidance of those who rely on cultural distinction for their position from popular cultural forms. At the same time, those who derive their capital from other fields are able to freely associate with popular cultural productions (which are stratified fields) while eschewing practices (which are un-stratified habitus within the cultural field). Capital conversion is not a mechanism of exchange conversion, but network integration. Understanding it as integration becomes theoretically generative.     

\subsection{Limitations on Positional Agency}
One problem with the above measure is that is creates the spectre of limitless positional agency: if a connection is mutually distinction increasing, actors should pursue it. Given that distinction increasing positional changes are possible, but go un-enacted, there must be constraints to distinction maximizing action. Theoretical considerations must be made to account for costs that constrict connection formation for distinction optimization. 

First, there is the matter of social foci \citep{feld1981focused}, or the physical and social spaces that create co-location for the formation of connections as an \textit{opportunity structure.} Brokers are particularly effective when they connect different clusters that would not otherwise be connected (Burt 1992), yet their advantaged position is otherwise threatened if the clusters begin to more directly connect. If the distance between them is maintained by the lack of overlap in their social foci, then re-positionings are foreclosed.   

Second, there is the problem of \textit{bargaining power}. The losers of positional change can retaliate by withdrawing their own connection to the changed node. This is particularly effective when multiple nodes withdraw their support in conjunction, with the retaliation leading the changed node to suffer a net decrease in network distinction. This runs counter to the predictions that would otherwise emerge from principles of triadic balance, as the maintenance of relational fragmentation is a legitimate strategy for producing high social distinction. Those who have experienced secondary education may be able to relate. Under conditions of fragmentation, where sides must be chosen, there is less of an issue of counter retaliation.

Third is the matter of \textit{cognitive load}. Humans have a limited cognitive capacity and time to successfully maintain the commitments requisite of social relationships \citep{dunbar1995neocortex}. While data may capture the structure of a specific field, people are part of many fields that remain unobserved in many empirical studies, but take away resources on their ability to act within the field of concern.

Fourth, is the \textit{cross linkages of fields} in that re-positionings in one field can come at a cost in another field because while the fields are separated in their boundaries, they are interconnected at the level of the individual. The formation of a business connection with someone could create positional ramifications in the field of politics for example, and so positional change is otherwise unenacted. For example, in the case of the Medici, the economic ties that they formed with the new families had political implications that otherwise prevented the old families from forming similar connections because it would hurt their standing. It was only the Medici's earlier fall from grace that enabled such a toleration \textit{padgett1993robust}.  

Fifth is \textit{positional uncertainty} about another's position in the field. While people are remarkably adept at inferring position in a field through the observation of habitus and capital resources, there remains a good deal of uncertainty about how new connections may upset existing competition structures, particularly in environments that are more circumscribed such as schools or organizations.     

Sixth, differences in \textit{habitus} mean that potential connections can fail to form because of the impacts it would have on their overall portfolio of positions. Just because of a potential gain in one area does not mean that it would not have costs in other areas that are unobserved. This is the problem above where capital conversion is not quite conversion, but incorporation.  

While it is difficult to build such impediments to positional agency into the distinction measure, study designs can carefully control for the problems they may otherwise impose. Studies of societal conflict which aggregates across fields and within constrained social units can significantly alleviate limitations on positional agency that are external to the distinction measure.  

\section{Conclusion}
Above we have introduced a way of constructing a general measure of distinction from network structure. The measure theorizes social distinction as a joint function of status and constraint derived from the pattern of network relations. 

In considering how social actors make decisions to maximize their social distinction within a network, scholars must consider the network counterfactuals that actors themselves weigh. In the case of the Medici network, actors considered their position within a network both with the Medici and without the Medici, and those who would increase their social distinction in the Medici’s absence undertook action to remove them. In the case of the karate network, the decision of the network to split was not made by Node 9, but 9 was forced to decide which pattern of relations to maintain. Imagining Node 9 in each of the two networks, 9’s social distinction was higher in Mr. Hi’s group than in the officer’s group, and so 9 chose this more desirable place in the social world. 

None of this is to suggest that actors are able to do the macro-level network calculation of distinction for the entire network when making decisions about their pattern of relations. However, humans are remarkable good at inputting in the network picture despite lacking the particulars and intuiting their position and optimal moves. Exploring the cognitive shortcuts that people use to map the complex details of their surrounding networks is an area ripe for academic inquiry. In addition, there is a future where computer modeling can estimate network evolution through individual maximization. 

This paper simply seeks to establish a measure of social distinction within a network structure. While the examples have focused on social distinction in shaping group politics, it is hoped that the measure proposed can be applied to and expand understandings of all forms of social distinction. Within sociology, the measure may prove particularly useful for developing new theory and explanations for shifts in competition over cultural distinction. However, use cases are anticipated to expand beyond the social world. The benefit of the measure maintains the rank of the highest status nodes within a system, but elevates those nodes that derive their position from non-redundant nomination sources. In the social world of power, politics, and alliances, it rewards the independence of one's support. In a task such as ordering search results, it increases the diversity of information that would be selected in the generation of a ranked list, eliminating redundant information from central cliques of high-status actors. The measure may hopefully find a wide variety of future applications. 


\newpage


\newpage
\bibliography{networks}

\bibliographystyle{apalike}

\newpage
\section*{Appendix: Distinction
Centrality Function} \label{sec:appendix}
\begin{lstlisting}[language=R]
distinction <- function(x) { 
   library(igraph) #calling igraph
   has.labels <- as.numeric(is.null(V(x)$name) == FALSE) #checking for vertex labels
   V <- vcount(x)
   if (has.labels == 0) {
      V(x)$name <- 1:V
      names <- V(x)$name 
      }
   if (has.labels == 1) {
      names <- V(x)$name 
      V(x)$name <- 1:V
      }
   s <- eigen_centrality(x)$vector
   d <- rep(V, 0) #initializing distinction vector
   u <- rep(V, 0) #initializing constraint vector
   for (i in as.character(V(x)$name)) {
      j <- names(neighbors(x, i))
      x.d <- delete_vertices(x, i)
      c <- as.numeric(is_connected(x.d) == TRUE) #checking for connectedness
      e <- as.numeric(ecount(x.d) == 0) #checking for empty
      if (c == 1 & e == 0) { #connected non-empty graph
         s.a <- eigen_centrality(x.d)$vector[j]
         } #end if
      else if (c == 0 & e == 1) { #empty graph
         s.a <- 0
         } #end first else 
      else if (c == 0 & e == 0) { #disconnected non-empty graph
         C <- components(x.d)$membership
         names(C) <- V(x)$name[-as.numeric(i)]
         s.a <- rep(0, length(C))
         for (k in unique(C)) {
            sub.g <- subgraph(x.d, names(which(C == k)))
            if (vcount(sub.g) > 1) {
               s.a[which(C == k)] <- eigen_centrality(sub.g)$vector
               }
            }
         names(s.a) <- names(C)
         s.a <- s.a[j]
         } #end second else
      u[i] <- sum(s.a)/length(s.a) #i's average neighbor centrality in node deleted subgraph
      d[i] <- s[i] - u[i] #i's distinction centrality
   } #end i for loop
   d[is.na(d)] <- 0
   u[is.na(u)] <- 0
   scalar <- mean(u)/mean(s)
   sd <- (s*scalar) - u
   dat <- data.frame(d = d, sd = sd, s = s, u = u, scalar = scalar)
   dat <- round(dat, 4)
   dat <- data.frame(n = names, dat)
   rownames(dat) <- 1:nrow(dat)
   return(dat)
} #end function
\end{lstlisting}

\end{document}